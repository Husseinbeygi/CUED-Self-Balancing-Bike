\documentclass[]{article}
\usepackage[margin=1in,top=0.5in]{geometry}
\usepackage{graphicx}
\usepackage{amsmath}
\usepackage{float}


\begin{document}



\title{\hrule \vspace{0.3cm} \textbf{James Dyson Foundation Undergraduate Bursary} \\ \large{Supporting IIB projects in problem solving and design} \vspace{0.2cm} \hrule }
\author{\textit{Philip Salmony, Wolfson College, pms67@cam.ac.uk}}
\date{7 November 2018}
\maketitle

\section{Applicant Information}
\begin{itemize}
\item Student name: \textit{Philip Salmony}
\item Student CRSID: \textit{pms67}
\item Supervisor's name: \textit{Dr. Fulvio Forni}
\item Supervisor's CRSID: \textit{ff286}
\item Project code: \textit{F-FF286-2}
\item Project title: \textit{Self-Balancing Bike}
\end{itemize}

\section{Questions to Applicants}

\subsection{Why did you choose to study engineering at university?}

I was set on engineering for quite a while before I finally started the course at university. During most of my teenage years I would be making and building things after school and on weekends. For instance, amplifiers for my electric guitar, robots, and various other devices - mainly from scratch. One could say that my choice was influenced by the fact that my dad is also an engineer, so often we would undertake projects together and he taught me programming when I was about 10 years old. Putting all this together, it seemed to me that engineering would be the perfect subject to choose at university. And being in my final year now, I must say I don't regret it one bit. Engineering to me is as fascinating now as it was when I was younger and I'm looking forward to pursuing it and learning more for a very long time to come.

\subsection{Where would you like to be in 10 years?}

In terms of my career and if everything goes to plan, I see myself working in engineering but not in exactly one particular field. Ideally, as my course is now, I will still be pursuing a variety of different engineering disciplines. I would love to be in a line of work that requires a mix of control, electrical, and information engineering. I personally find aerospace to be extremely interesting and I pursue this area actively in my free time, for instance having co-founded the Cambridge 	University Unmanned Air Systems Society, where we design, build, and fly completely autonomous fixed-wing aircraft. Doing something similar in a job - or in that sector - would definitely be a dream, even though I will always pursue this 	in my free time as well. On the personal side, I hope that I will still be pursuing the hobbies I have today - both engineering-related and not, preferably back in my home country of Germany.

\subsection{What invention do you wish you had thought of?}

The jet engine. For me this is one of the greatest inventions of humanity. The amount of thought and effort that goes into the design and production of a jet engine is incredible. For instance, the fact that the turbine blades are operating at a temperature above their melting point, or that they are grown from a single crystal. It is amazing that with all the high pressures, high temperatures, thousands of different parts, and so on, modern jet engines are incredibly safe and enable millions of passengers per year to be flown across the whole world in just a couple of hours.

\subsection{Who do you look up to?}

No particular single person comes to mind, but rather a certain type of person. I	really look up to people that can not only come up with a brilliant idea, but also develop that idea further, and then bring it to life and not give up. A lot of times I find people have a great spur of motivation when starting something such as a project, but then fail to complete it due to distractions or such. Therefore, I admire people who can finish what they have started with determination throughout.

\subsection{What advice would you give a young person considering engineering as a career?}

If you are set on engineering, take a fair amount of time to consider which engineering discipline you would like to pursue and actively seek out what you would be studying in that subject. Unfortunately, only a small amount of university courses worldwide offer a 'general' first year or two, where you can get a feel for different types of engineering areas. Luckily, Cambridge offers this which is a real plus. I originally applied to do Mechanical Engineering, but now I'm specialising in Electrical and Control - so for me that would have been a sub-optimal choice.

\subsection{Engineering is... (in one sentence)}

... taking relevant ideas from maths and physics, combining them, and producing something useful to solve a real-world problem for society.

\subsection{My hidden talent is...}

... playing the electric guitar. I've been playing in various bands (rock, big-band, 	etc.) both in Germany and the UK for about 10 years now. In particular, music from	the 70s and 80s, such as bands like Queen, Van Halen, and Guns 'n' Roses.

\section{Project Information}

\subsection{What will the project involve?}

The project involves modelling, simulation, and control system design for an actuated, rider-less bicycle. Furthermore, a full-scale bicycle will then be fitted with actuators, sensors, and a control unit, to test the theoretical work on a real-world, practical system.

\subsection{Why did I choose this project?}

Nearly everyone has ridden a bicycle once before in their lives and everyone eventually learns to ride it after a couple of tries. Additionally, we know that a rider-less bicycle is actually self-stable above a certain speed. However, people have not been entirely successful in giving a good explanation as to why that is. This poses several interesting questions, for example:

\begin{itemize}
\item How does a human stabilise a bicycle and can we replicate this behaviour with motors? 
\item What makes a bicycle more or less stable (such as changes in geometry, mass distributions, wheel size, etc.)?
\item Would there be a benefit in having an 'assisted' bicycle for certain people and situations?
\end{itemize}

\noindent I therefore chose this project to try and answer - at least in part - some of these questions and additionally because there are so many engineering aspects to it. From control system design, to mechanical implementation, and to gaining an understanding of how a bicycle stays self-stable. It is amazing how much feedback control is a part of most engineering disciplines, from cruise control in cars, to flight control systems, and then to self-stabilising bikes. I feel that undertaking this project will greatly improve my understanding in a variety of subjects and finally: building and testing an actuated bike will be a whole load of fun!

\subsection{How would the bursary be used?}

The bursary would be mainly used to allow the development of a full-scale, actuated bicycle. This includes items such as motors, microcontrollers, sensors, 3D printing filament for smaller parts, and various other components. A bursary would also allow the development of an initial scale model, to test the control systems before moving on to a rather larger, full-scale version. It would be great to be able to do both and show that a full-scale 'autonomous' bicycle is possible.

\subsection{Which outreach materials will be produced?}

I will build a prototype of a rider-less bike that does not fall and drives itself along a predefined path. The bike will carry loads and will keep its balance in presence of disturbances, through sensor feedback. Outreach material will be produced through videos and pictures of the experimental activity. The actual working prototype can also be used as outreach materials: a rider-less, self-balancing bike is an excellent way to inspire young adults and kids. It is hard to imagine a more effective way to illustrate the excellent, passionate work of an engineer than showing an actual bike that is able to balance and drive itself through a clever combination of sensors, actuators, and control algorithms.

\end{document}